\begin{thebibliography}{99}
    
\bibitem{rubiano} Rubiano, G. (2020). \textit{Iteración y fractales (con mathematica ®)}. Universidad Nacional de Colombia.

\bibitem{Dreher} F. Dreher; T. Samuel. (2014). Continuous Images of Cantor’s Ternary Set. \textit{The American Mathematical Monthly}, 121(7), 640–643. \url{https://doi.org/10.4169/amer.math.monthly.121.07.640}.

\bibitem{Benyamini} Benyamini, Y. (1998). Applications of the Universal Surjectivity of the Cantor Set. \textit{American Mathematical Monthly}, 105, 832-839. \url{https://arxiv.org/pdf/1303.3810.pdf}.

\bibitem{alma991013963162704990} Edgar, G.A. (2008). \textit{Measure, Topology, and Fractal Geometry} (2nd ed. 2008.). Springer New York. \url{https://doi.org/10.1007/978-0-387-74749-1}
\bibitem{alma991007242979704990} Mandelbrot, B. (1983). \textit{The Fractal geometry of nature}. Freeman.
\bibitem{alma991007022459704990} Falconer, K. (1990). \textit{Fractal geometry: mathematical foundations and applications}. John Wiley.
\bibitem{Hausdorff1919} Hausdorff, F. \textit{Dimension und ausseres Mass.} Mathematische Annalen 79 (1919): 157-179. \url{http://eudml.org/doc/158784}.
\bibitem{Hurewicz-Wallman}Hurewicz, W.; Wallman, H. (1948). \textit{Dimension Theory}. Princeton University Press.
\bibitem{Moran}Moran, P.A. (1946). \textit{Additive functions of intervals and Hausdorff measure}. In \textit{Mathematical Proceedings of the Cambridge Philosophical Society} (Vol. 42, No. 1, pp. 15-23). Cambridge University Press.
\bibitem{Wagon}Wagon, S. (2010). \textit{Mathematica in Action}. Springer Publishing. p. 223
\bibitem{Apuntes-AMI-Paya} Payá, R (2008). \textit{Apuntes de Análisis Matemático I}...
\bibitem{Ostrowski} Ostrowski, A.M. (1973). \textit{Solution of equations in Euclidean and Banach spaces.} (3rd ed.). Academic Press.
\bibitem{Atkinson} Atkinson, K; Han, W. (2009). \textit{Theoretical Numerical Analysis: A Functional Analysis Framework} (3rd ed. 2009). Springer New York. \url{https://doi.org/10.1007/978-1-4419-0458-4}
\bibitem{Dubeau-Gnang} Dubeau, F.; Gnang, C. (2018). Fixed Point and Newton’s Methods in the Complex Plane. \textit{Journal of Complex Analysis, 2018}, 1–11. \url{https://doi.org/10.1155/2018/7289092}
\bibitem{John-Milnor} Milnor, J. (2006). \textit{Dynamics in one complex variable.} (3rd ed.). Princeton University Press. \url{https://doi.org/10.1515/9781400835539}
\bibitem{Barnsley} Barnsley, M; Rising, H. (1993). \textit{Fractals everywhere} (2nd ed.). Academic Press.
\bibitem{agness-scott} \textit{Pythagorean Tree}. (s. f.). Agness Scott College. \url{https://larryriddle.agnesscott.org/ifs/pythagorean/pythTree.htm}
\bibitem{Sandra-Snyder} Snyder, S.S. (2006). Fractals and the Collage Theorem. \textit{MAT Expository Papers}. 49. \url{https://digitalcommons.unl.edu/mathmidexppap/49}.
\bibitem{Evan-Dummit} Dummit, E. (2015). \textit{Dynamics, Chaos, and Fractals (part 4): Fractals}. Rochester MTH 215. \url{https://web.northeastern.edu/dummit/docs/dynamics_4_fractals.pdf}.

\bibitem{Bandt} Bandt, C., Nguyen Viet Hung, Rao, H. (2006). On the Open Set Condition for Self-Similar Fractals. \textit{Proceedings of the American Mathematical Society}, 134(5), 1369–1374. \url{http://www.jstor.org/stable/4097989}


\bibitem{Foroutan} Foroutan-pour, K., Dutilleul, P., Smith, D. (1999). Advances in the implementation of the box-counting method of fractal dimension estimation. \textit{Applied Mathematics and Computation, 105}(2–3), 195–210. \url{https://doi.org/10.1016/s0096-3003(98)10096-6}.

\bibitem{cgdirector} Glawion, A. (2022, 12 abril). \textit{CPU vs. GPU Rendering – What’s the difference and which should you choose?} CG Director. Recuperado 20 de mayo de 2022, de \url{https://www.cgdirector.com/cpu-vs-gpu-rendering/#:%7E:text=GPU%20Based%20Rendering%20And%20GPU%20Focused%20Render%20Engines,-GPU%20rendering%20is&text=For%20one%2C%20GPUs%20are%20much,unlike%20CPUs%20which%20operate%20serially}.

\bibitem{computer-graphics} Hughes, J.F., Van Dam, A., McGuire, M., Sklar, D.F., Foley, J.D., Feiner, S.K., Akeley,K. (2014). \textit{Computer graphics: principles and practice} (3rd ed.). Addison-Wesley.


\bibitem{MDN-1} \textit{Adding 2D content to a WebGL context - Web APIs | MDN}. (2022, 24 abril). MDN Web Docs. \url{https://developer.mozilla.org/en-US/docs/Web/API/WebGL_API/Tutorial/Adding_2D_content_to_a_WebGL_context}

\bibitem{wikipedia-webgl} colaboradores de Wikipedia. (2022, 16 febrero). \textit{WebGL}. Wikipedia, la enciclopedia libre. \url{https://es.wikipedia.org/wiki/WebGL}



\bibitem{MDN-2} \textit{Data in WebGL - Web APIs | MDN.} (2022, 14 marzo). MDN Web Docs. \url{https://developer.mozilla.org/en-US/docs/Web/API/WebGL_API/Data}

\bibitem{MDN-3} \textit{Getting started with WebGL - Web APIs | MDN.} (2022, 24 abril). MDN Web Docs. \url{https://developer.mozilla.org/en-US/docs/Web/API/WebGL_API/Tutorial/Getting_started_with_WebGL}

\bibitem{learn-opengl} \textit{LearnOpenGL - OpenGL.} (s. f.). OpenGL. \url{https://learnopengl.com/Getting-started/OpenGL}

\bibitem{khronos} \textit{OpenGL ES - The Standard for Embedded Accelerated 3D Graphics}. (2011, 19 julio). The Khronos Group. \url{https://www.khronos.org/api/opengles}

\bibitem{MDN-4} \textit{WebGL: 2D and 3D graphics for the web - Web APIs | MDN}. (2022, 27 abril). MDN Web Docs. \url{https://developer.mozilla.org/en-US/docs/Web/API/WebGL_API}

\bibitem{HTML5rocks} \textit{WebGL Fundamentals - HTML5 Rocks.} (s. f.). HTML5 Rocks - A Resource for Open Web HTML5 Developers. \url{https://www.html5rocks.com/en/tutorials/webgl/webgl_fundamentals/}

\bibitem{MDN-5} \textit{WebGL model view projection - Web APIs | MDN.} (2022, 26 abril). MDN Web Docs. \url{https://developer.mozilla.org/en-US/docs/Web/API/WebGL_API/WebGL_model_view_projection#clip_space}

\bibitem{renderingcontextdoc} \textit{WebGLRenderingContext - Web APIs | MDN}. (2022, 20 enero). WebGLRenderingContext. \url{https://developer.mozilla.org/en-US/docs/Web/API/WebGLRenderingContext}

\bibitem{RT-que-es} López, P. (2020, 30 abril). \textit{Ray Tracing: ¿Qué es y para qué sirve? - Definición}. GEEKNETIC. \url{https://www.geeknetic.es/Ray-Tracing/que-es-y-para-que-sirve}

\bibitem{Hart-1995} Hart, J. (1995). Sphere Tracing: A Geometric Method for the Antialiased Ray Tracing of Implicit Surfaces. \textit{The Visual Computer}. \url{https://doi.org/10.1007/s003710050084}

\bibitem{3D-SDFs} Quilez, I. (s. f.). \textit{3D SDFs}. Íñigo Quilez. Recuperado 14 de mayo de 2022, de \url{https://iquilezles.org/articles/distfunctions/}

\bibitem{quaternions} Conway, J. H., Smith, D. A. (2003). \textit{On quaternions and octonions}. 21. CRC Press LLC.

\bibitem{quaternion-product} Pharr, M., Jakob, W., Humphreys, G. (2017). \textit{Physically Based Rendering} (3rd ed.). 57-121. Morgan Kaufmann. \url{https://doi.org/10.1016/B978-0-12-800645-0.50002-6} 

\bibitem{keenan-crane} Crane, K. (2005). Ray Tracing Quaternion Julia Sets on the GPU. \textit{University of Illinois at Urbana-Champaign}

\bibitem{Hubbard-Douady} Douady, A., Hubbard, J. (2009). \textit{Exploring the Mandelbrot set. The Orsay Notes}.

\bibitem{normals-sdf} Quilez, I. (s. f.-b). \textit{Normals for an SDF}. Íñigo Quilez. Recuperado 15 de mayo de 2022, de \url{https://iquilezles.org/articles/normalsSDF/}

\bibitem{distance-fractals} Quilez, I. (2004). \textit{Distance to fractals}. Íñigo Quilez. Recuperado 16 de mayo de 2022, de \url{https://iquilezles.org/articles/distancefractals/}

\bibitem{mandelbub} Quilez, I. (2009). \textit{Mandelbub}. Íñigo Quilez. Recuperado 16 de mayo de 2022, de \url{https://iquilezles.org/articles/mandelbulb/}
\end{thebibliography}
