%%%%%%%%%%%%%%%%%%%%%%%%%%%%%%%%%%%%%%%%%%%%%%%%%%%
%% Capítulo 4: Sistemas de funciones iteradas
%%%%%%%%%%%%%%%%%%%%%%%%%%%%%%%%%%%%%%%%%%%%%%%%%%%

Como venimos viendo ya en los dos últimos capítulos, la iteración es una poderosa herramienta en la generación de imágenes fractales. Sin embargo, hasta ahora siempre nos estamos basando en buscar convergencia y velocidad de convergencia de sucesiones de iteradas de funciones complejas, las cuales evalúan un número complejo $z\in\C$ para devolver otro número complejo $f(z)\in\C$. Si recuperamos la identificación $z=x+y\cdot i\simeq (x,y)\in\R^2$ podemos ver las funciones complejas como campos vectoriales de $\R^2$, y si en lugar de aplicar una función a un único punto la aplicamos a un conjunto de puntos llegamos a la base de los \textit{Sistemas de Funciones Iteradas}, en adelante SFI. 

La matemática que explica los SFI se puede encontrar en el clásico libro \textit{Fractals Everywhere}\cite{Barnsley} de \textit{Michael Barnsley}. Por su parte, la geometría fractal nació en 1977 con la publicación de \textit{The fractal geometry of nature}\cite{alma991007242979704990} por parte de \textit{Benoit Mandelbrot}. En general, gracias a la geometría fractal y ayudándonos de los SFI podemos recrear resultados de imágenes y objetos con un nivel de detalle que la geometría euclídea no puede conseguir. Sin embargo, el problema inverso también es interesante: ¿es posible, a partir de un objeto, describirlo matemáticamente mediante SFI? Este es un área de la matemática aún abierta y en la que a día de hoy se continua trabajando. Uno de los resultados más conocidos en este ámbito es el \textit{teorema del collage}, del cual hablaremos más adelante. %TODO referenciar teorema y sección

\section{Transformaciones afines en el plano euclídeo}

Si nos restringimos al plano euclídeo $\R^2$ visto como espacio afín, previo a la definición formal de SFI necesitamos unas nociones sobre transformaciones afines y maneras de representarlas, ya que estas serán las que compongan fundalmente los SFI.

\begin{definicion}[Transformación afín]
    Una transformación $w:\R^2\longrightarrow\R^2$ de la forma
    \begin{equation}
        w(x_1,x_2)=(ax_1+bx_2+e,cx_1+dx_2+f) \ \ \forall (x_1,x_2)\in\R^2
    \end{equation}
    donde las constantes $a,b,c,d,e,f$ son números reales es denominada una \textbf{transformación afín} del plano euclídeo.
\end{definicion}

Una forma equivalente de denotar $w$ matricialmente es tomando la matriz $A=\begin{pmatrix}
    a & b \\
    c & d
\end{pmatrix}\in\mathcal{M}_2(\R)$ y el vector $b=\begin{pmatrix}
    e \\
    f
\end{pmatrix}$ y expresar $w$ como:

\begin{equation}
    w(x) =
    w\begin{pmatrix}
        x_1 \\
        x_2
    \end{pmatrix} = Ax + b = \begin{pmatrix}
        a & b \\
        c & d
    \end{pmatrix}\begin{pmatrix}
        x_1 \\
        x_2
    \end{pmatrix}+\begin{pmatrix}
        e \\
        f
    \end{pmatrix} \ \ \forall \ x = \begin{pmatrix}
        x_1 \\
        x_2
    \end{pmatrix}\in\R^2.
\end{equation}


